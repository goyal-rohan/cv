\documentclass[margin,line, 10pt]{res}
\usepackage{hyperref}
\usepackage{url}
\usepackage{fontawesome}
\usepackage[dvipsnames]{xcolor}
\definecolor{hypercolor}{HTML}{800000}
\definecolor{top}{HTML}{ff0000} 
\usepackage{hyperref}
\hypersetup{
    colorlinks=true,
    urlcolor=hypercolor,
}

\usepackage{fancyhdr}
\pagestyle{fancy}
\renewcommand{\headrulewidth}{0pt}
\renewcommand{\footrulewidth}{0pt}
\fancypagestyle{lastpage}
{
    \fancyfoot[L] {\sffamily \thepage}    
    \fancyfoot[R] {\sffamily{Last update on \today}}
}
\fancyhead{}
\fancyfoot{}
\fancyfoot[L]{\sffamily \thepage}
\usepackage{mathpazo} 
\usepackage{eulervm}
% \usepackage[default]{sourcesanspro}
% \usepackage[T1]{fontenc}

% \usepackage{fontspec}
% \setmainfont[
% BoldFont = Source Sans Pro-{Semi-bold}]{Source Sans Pro}
% \setmainfont{Caladea}

\oddsidemargin -.5in
\evensidemargin -.5in
\textwidth=6.0in
\itemsep=0in
\parsep=0in
% if using pdflatex:
%\setlength{\pdfpagewidth}{\paperwidFth}
%\setlength{\pdfpageheight}{\paperheight} 

\newenvironment{list1}{
  \begin{list}{\ding{113}}{%
      \setlength{\itemsep}{0in}
      \setlength{\parsep}{0in} \setlength{\parskip}{0in}
      \setlength{\topsep}{0in} \setlength{\partopsep}{0in} 
      \setlength{\leftmargin}{0.17in}}}{\end{list}}
\newenvironment{list2}{
  \begin{list}{$\bullet$}{%
      \setlength{\itemsep}{0in}
      \setlength{\parsep}{0in} \setlength{\parskip}{0in}
      \setlength{\topsep}{0in} \setlength{\partopsep}{0in} 
      \setlength{\leftmargin}{0.2in}}}{\end{list}}

\begin{document}

\name{Rohan Goyal \vspace*{.1in}} 

\begin{resume}
\section{\sc Contact Information}
Chennai Mathematical Institute \hfill {rohang@cmi.ac.in}\\
H1 SIPCOT IT Park, Siruseri, Chennai \hfill \href{https://goyal-rohan.github.io/}{\texttt www.goyal-rohan.github.io}

\section{\sc Research Interests}
I am interested in theoretical computer science broadly. My main interests lie in approximation algorithms and hardness of approximation, coding theory, combinatorics, expanders and applications, PCPs and pseudorandomness along with applications to other areas like cryptography.
%----------------------------------------------------------------------------------------
%	EDUCATION 
%----------------------------------------------------------------------------------------
\section{\sc Education}


{\bf Chennai Mathematical Institute}, Chennai, India \hfill September 2021-April 2024

\vspace{-0.4cm}
B.Sc.(Honours) in Mathematics and Computer Science\\ {\bf CGPA:} 9.55/10.00, {\bf CS GPA:} 10.0/10.0 \hfill July 2023


\vspace*{-0.3cm}
Detailed Coursework and Courses TAed towards the end of the document.


%----------------------------------------------------------------------------------------
%	HONORS
%----------------------------------------------------------------------------------------
\section{\sc Honors and Awards} 
Deputy Leader India, {\bf European Girls Mathematics Olympiad 2023} \href{https://www.egmo.org/egmos/egmo12/countries/country35/}{Indian team} \hfill 2022

\vspace*{-2.5mm}
Bronze Medal at {\bf International Mathematical Olympiad (IND1)} \hfill 2021

\vspace*{-2.5mm}
Iranian Geometry Olympiad Advanced 2021: Silver Ruler, Advanced Category, 1st in India \hfill 2021

\vspace*{-2.5mm}
Asian Pacific Mathematical Olympiad, Bronze Medal \hfill 2020

\vspace*{-2.5mm}
Indian National Mathematical Olympiad: Qualified for National Camp\\ India Rank 2 in 2021 \hfill 2020, 2021

\vspace*{-2.5mm}
International Collegiate Programming Competition Regionals \hfill 2023
 
% \hspace{-2em} \faTrophy \hspace{.5em}
\vspace*{-2.5mm}
Simon Marais Mathematics Competition: Topped CMI, Top 20 overall \hfill 2022

\vspace*{-2.5mm}
Kishore Vigyanik Pratyogita Yojana, All India Rank 87 and am receiving INSPIRE scholarship \hfill 2020

\vspace*{-2.5mm}
Sriram Scholarship: Complete tuition fee waiver for attending CMI\hfill 2021-2024

\vspace*{-2.5mm}
Asian Pacific Linguistics Olympiad, Unofficial Honourable Mention \hfill 2021

\vspace*{-2.5mm}
Qualified Panini Linguistics Olympiad and wrote Indian team selection tests \hfill{2021}

\vspace*{-2.5mm}
Participated in Virtual Maths Beyond Limits Camp 2020 usually conducted in Poland. I also returned as a volunteer in 2021 for Olympic Training. \hfill 2020

\vspace*{-2.5mm}
Qualified Zonal Informatics and Computing Olympiads \hfill 2020

\vspace*{-2.5mm}
Qualified National Talent Search Examination and was awarded an NTSE scholarship. \hfill 2019-20
%----------------------------------------------------------------------------------------
%	RESEARCH    
%----------------------------------------------------------------------------------------
\section{\sc Internships, Research Projects}
{\bf Tata Institute of Fundamental Research}, Navy Nagar, Mumbai, India

\vspace{-.3cm}
{\em Intern} \hfill {May 2023 - present, (concluded August 2023 officially)}\\
Worked under \href{https://www.tifr.res.in/~prahladh/}{Professor Prahladh Harsha}.

\vspace*{.05in}  
\begin{list2}
\item Read Anup Rao's Sunflowers: Soil to Oil paper and presented recent results on the Sunflower Conjecture. (paused work on this project)
\item Looked at some problems in secret sharing and some algorithms (with Prof. Akshayram and Rathnakar).
\item Read and presented Amnon Ta-Shma's paper on local testability of multiplicity codes.
\item Read notes and papers on high dimensional expanders and sampling algorithms using HDXs.
\item Read notes, and DELLM21 paper on CCC codes (constant rate, distance and locality) and tried to improve bounds. (ongoing work)
\item Discovered a near linear time algorithm for list decoding multiplicity codes (ongoing work, with Prahladh Harsha, Mrinal Kumar and Ashutosh Shankar, resulted in a paper)
\end{list2}

\eject
{\bf IIT Bombay}

\vspace{-.3cm}
{\em Reading Project} \hfill {May 2022 - July 2022}\\
Worked under \href{https://www.cse.iitb.ac.in/~rgurjar/}{Professor Rohit Gurjar}.

\vspace*{.05in}  
\begin{list2}
\item Read Michael Kim's lecture notes and Nitin Saxena's 2009 survey on polynomial identity testing. I also looked at some ROABPs and other methods of computation and depth reduction techniques.
\end{list2}

{\bf CMI}

\vspace{-.3cm}
{\em Research Internship (Planned)} \hfill {January 2024 - }\\
Working on problems in matchings and fair division with Prof. Prajakta Nimbhorkar. 
%--------------------------------------------------------------------------------
%    Publications
%-----------------------------------------------------------------------------------------------
\section{\sc Writing and Publications}

%G., Prahladh Harsha, Mrinal Kumar, Ashutosh Shankar, 2023. Near Linear Time for List Decoding Multiplicity Codes. \hfill \href{}{arxiv} \href{}{eccc} 

\vspace{-.1cm}

{\bf Preprints}
\begin{list2}
    \item \textit{Fast list-decoding of univariate multiplicity and folded Reed-Solomon codes}\\
     with Prahladh Harsha, Mrinal Kumar and Ashutosh Shankar.
\end{list2}


{\bf Olympiad Writing}

I have written many articles and handouts during the time I was preparing for Mathematical Olympiads. Most of them can be found on \href{https://rgtdfg.blogspot.com/p/handouts.html}{my blog}. Some selected ones are:
\begin{list2}
    \item Polynomials, February 2021 \hfill [\href{https://www.dropbox.com/s/yo31nat6z5ggaue/Polynomials.pdf?dl=0}{pdf}]
    \item Chip Firing Games, September 2020 \hfill [\href{https://www.dropbox.com/s/66a3xw6xad35i8y/chip_firing_presentation.pdf?dl=0}{slides}] [\href{https://www.dropbox.com/s/fu0xmn8u42qdhyl/Chip%20Firing.pdf?dl=0}{pdf}]
    \item Combinatorial Nullstellensatz, May 2020 \hfill [\href{https://www.dropbox.com/s/9vjbqeec17hubov/Combo-Null.pdf?dl=0}{pdf}]
    \item Circumcircle-Excircle Configuration, October 2020 \hfill [\href{https://www.dropbox.com/s/qtxwbpe6tffyyo0/circumex.pdf?dl=0}{pdf}]
\end{list2}


%--------------------------------------------------------------------------------
%    Talks
%-----------------------------------------------------------------------------------------------
\section{\sc Talks and Presentations}


%{\bf }, CMI Theory Seminar \hfill November 2023

%\vspace*{-2.5mm}
{\bf Sunflower Conjecture}, CMI Student Seminar \hfill Sep, 2023

\vspace*{-2.5mm}
{\bf Combinatorial Nullstellensatz} TIFR STCS Student Seminar \hfill Aug, 2023

\vspace*{-2.5mm}
{\bf Games on Graphs with Imperfect Information}, Games on Graphs Course \hfill May, 2023

\vspace*{-2.5mm}
{\bf Saks-Zhou 1999, $BP_HSPACE(S)\subset DSPACE(S^{3/2})$}, Complexity 2 Course \hspace{0.5em} [\href{https://www.dropbox.com/scl/fi/d3o6bhib7i7ifok6b8t79/Saks-and-Zhou-1999-Report-Aryan-Agarwala-and-Rohan-Goyal.pdf?rlkey=1xirwrhdz8541vuiq9l189apk&dl=0}{Report}] \hfill Dec, 2022 

%--------------------------------------------------------------------------------
%    Conferences and Workshops attended
%-----------------------------------------------------------------------------------------------
\section{\sc Conferences and Workshops attended}

\begin{list2}
    \item Workshop on Algebra and Computation 2023, attended via Zoom
    \item STCS Day 2023, TIFR 
    \item Chennai-Tirupati Intercity Number Theory Conference 2023
    \item FSTTCS 2022, Pre Conference workshop on "Algorithms under Uncertainty".
    \item FSTTCS 2022
\end{list2}

%--------------------------------------------------------------------------------
%    PROJECTS
%-----------------------------------------------------------------------------------------------
\section{\sc Olympiad Projects and Outreach}
%--Matholy--
{\bf Indian Mathematical Olympiad Program} \hfill 2021-present\\
I am involved in various roles in the Indian Mathematical Olympiad Program. Some of these are:
\begin{list2}
    \item Deputy leader for India at EGMO 2023
    \item Paper setting, grading, problem proposing: IMO TSTs: 2023$\mid$ EGMO TSTs 2021, 2022, 2023$\mid$  INMO 2023, 2024
    \item Teaching, training: EGMOTC 2022, 2023$\mid$IMOTC 2023$\mid$INMOTCs 2022, 2023$\mid$EGMO PDC 2022, 2023$\mid$IMO PDC 2022, 2023
\end{list2}

%--IFOMG--
{\bf Championship of Mathematical and Logical Games} {\hspace{1em} [\href{https://www.linkedin.com/posts/ghislainfourny_hamen-bahut-bahut-badhia-laga-ki-pahle-baar-activity-6969912358307536896--pLr?utm_source=share&utm_medium=member_desktop}{Post by Ghislain Fourny, ETH}]} \hfill 2022\\
Conducted the qualifying, regional and national stages of the Championship of Mathematical and Logical Games in India for the first time. We had over 3000 participants and took 16 students across to EPFL, Switzerland for the international finals. We did not charge any participants at any stage and were completely sponsored by Inshorts Ltd. Organized under "Indian Federation of Mathematical Games" which I co-founded. 

%--Sophie Fellowship--
{\bf Sophie Fellowship}{\hspace{1em} [\href{https://www.sophiefellowship.in/}{Web}]}  \hfill 2021-present\\
Co-founded Sophie Fellowship to provide more resources and guidance to talented students interested in participating in international and national mathematical Olympiads. The team consists of various IMO and EGMO medalists. I am now not actively teaching, and am only in an advisory role due to my increased involvement with the official Indian program. I have also taught many of the sessions.

%--OMath--
{\textbf {Online Math Club}}{ \hspace{1em} [\href{https://omath.club/}{Web and Blog}] \hspace{0.4em} [\href{https://www.youtube.com/@OMath/videos}{YouTube}]} \hfill 2021-Present\\
I was the co-director for Online Math Club (OMC) and am currently in an advisory role. We take weekly public lectures on topics in mathematics (Olympiad and introductory college material) accessible to high schoolers and share them on YouTube later. We have collected over 100 lectures on various different topics in mathematics.  I have also taught many of these sessions.

%--STEMS--
{\bf STEMS}{\hspace{1em} [\href{https://www.tessellate.cmi.ac.in/stems}{Web}]\hspace{0.4em} [\href{https://www.youtube.com/@TessellateCMI}{YouTube}]} \hfill 2021-Present\\
At CMI, we annually conduct exams in mathematics, computer science and physics for high school students and undergraduates. We then call the highest performing students to CMI for a 3-day camp where we have many renowned speakers in the three subjects. I was the math head in 2021, overall head in 2022 and am in an advisory role this year.

%--RUC--
{\bf Unofficial IMOTC and CAMP}{\hspace{1em} [\href{https://sites.google.com/view/campamp/home}{Web}]\hspace{0.4em} [\href{https://www.youtube.com/@campforadvancedmathematica4861}{YouTube}]} \hfill 2020-2021 \\
We unofficially organized online camps in 2020 and 2021 for students qualified in the national Olympiad camp and some other students due to the cancellation of the official program due to CoViD situation. I took over 20 lectures in the two camps and many lectures were uploaded on YouTube. Students continued the program into 2022 and conducted it with our guidance. I have also taught many of the sessions.

%--OTIS--
{\bf Individual Instruction} \hspace{1em} [\href{https://web.evanchen.cc/otis.html}{OTIS web}]  \hfill 2020-Present\\
I take lectures for mathematical Olympiads for some students from across the world mostly through OTIS. I have been fortunate to work with extremely talented students. Many of them have performed well in contests as well and qualified for the national camps and teams for IMO and EGMO.

%--Misc--
{\bf Miscelleneous Talks and Activities}\\
I often take some talks on Olympiads or elementary topics. For example, I have taken sessions on elementary number theory for Mathematical Initiatives in Nepal and for Informatics Olympiad training in India via CodeChef and Unacademy. I have also taken some talks on triangle inequality, parity and such for students from 5th to 8th grade for DhiMath, RAM and other organizations. 


%--------------------------------------------------------------------------------
%    CMI EXPERIENCE
%-----------------------------------------------------------------------------------------------
\section{\sc Coursework}

{\bf Mathematics Core}
\begin{list2}
    \item Algebra 1 (Linear Algebra) \hfill Semester 1, Fall 2021 
    \item Algebra 2 (Group Theory) \hfill Semester 2, Spring 2022
    \item Algebra 3 (Rings and Field Theory) \hfill Semester 3, Fall 2022
    \item Analysis 1 (Analysis on $\mathbb{R}$) \hfill Semester 1, Fall 2021
    \item Analysis 2 (Analysis on $\mathbb{R}^n$) \hfill Semester 2, Spring 2022
    \item Analysis 3 (Generalized Metric Spaces and Fourier Analysis) \hfill Semester 3, Fall 2022
    \item Calculus (Calculus on Manifolds) \hfill Semester 3, Fall 2022
    \item Complex Analysis \hfill Semester 4, Spring 2023
    \item Differential Equations \hfill Semester 4, Spring 2023
    \item Topology (Point-set and Algebraic Topology) \hfill Semester 4, Spring 2023
    \item Introduction to Probability Theory \hfill Semester 2, Spring 2022
\end{list2}

{\bf Computer Science Core}
\begin{list2}
    \item Introduction to Programming in Haskell \hfill Semester 1, Fall 2021
    \item Advanced Programming in Python \hfill Semester 2, Spring 2022
    \item Discrete Mathematics \hfill Semester 2, Spring 2022
    \item Design and Analysis of Algorithms \hfill Semester 3, Fall 2022
    \item Theory of Computation \hfill Semester 3, Fall 2022
    \item Programming Language Concepts (Concurrency, Lambda Calculus) \hfill Semester 4, Spring 2023
\end{list2}

{\bf Computer Science Graduate Electives}
\begin{list2}
    \item Complexity Theory 1 \hspace{0.4em} [\href{https://www.cmi.ac.in/~prajakta/courses/s2022/index.html}{Course-Page}]\hfill Semester 2, Spring 2022
    \item Complexity Theory 2 (Pseudorandomness and PCPs)\hfill Semester 3, Fall 2022
    \item Games on Graphs \hspace{0.4em} [\href{https://www.cmi.ac.in/~sri/Courses/GGRP/2023/index.html}{Course-Page}]\hfill Semester 4, Spring 2023
    \item Algebra and Computation (Polynomial and Group Computation) \hfill Semester 4, Spring 2023
    \item Timed Automata \hspace{0.4em} [\href{https://www.cmi.ac.in/~sri/Courses/TA/2023/index.html}{Course-Page}] \hfill Semester 5, Fall 2023
    \item Advanced Algorithms \hspace{0.4em} [\href{https://www.cmi.ac.in/~prajakta/courses/f2023/index.html}{Course-Page}] \hspace{0.1cm} [\href{https://drive.google.com/file/d/1cAc0n1nwUdwJThporB_nBtJbyXIRuygb/view?usp=drivesdk}{my Class-Notes}]\hfill Semester 5, Fall 2023
    \item Algorithmic Coding Theory 2 (half semester) \hfill Semester 5, Fall 2023
    \item Combinatorial Optimization (planned) \hfill Semester 6, Spring 2024
\end{list2}

\section{\sc {\bf TA}}\label{sec: TA}
Courses TAed:
\begin{list2}
    \item Complexity Theory 1 \hfill Semester 4, Spring 2023
    \item Discrete Mathematics \hfill Semester 4, Spring 2023
    \item Theory of Computation \hfill Semester 5, Fall 2023
    \item Discrete Mathematics \hfill Semester 6, Spring 2024
    \item Expander Graphs and applications \hfill Semester 6, Spring 2024
\end{list2}


\section{\sc Miscellaneous}

{\bf A happy and proud moment:} A student I worked with actively joined MIT as an undergraduate in fall 2023. I then received an extremely kind congratulatory letter from the MIT admissions department saying that I had been named by the student as an especially influential teacher in his development! \hfill [\href{https://www.dropbox.com/scl/fi/3qgucmmeimm4omc644ikz/mit_atul.jpg?rlkey=mflk16ps5md13whz80o50t2yj&dl=0}{The letter}]

\vspace*{0.1mm}
{\bf Languages:} Haskell, Python, Java, C++, \LaTeX, SageMath

\vspace*{0.1mm}
{\bf Hobbies}
\begin{list2}
    \item Drama Club, CMI: I regularly attend Drama Club meetings in CMI.
    \item Chess: I used to participate in chess tournaments actively in middle school and am FIDE rated. Currently, I mostly play bullet and blitz chess online and solve puzzles.
    \item Badminton and Go: I play badminton regularly. I was also actively learning Go and intend to restart playing.
    \item Music: I love all kinds of music (except metal) and always appreciate recommendations!
    \item Film and TV: Same as music! (Change metal to horror.) Always appreciate recommendations!
    \item Potterhead: I am a Hufflepuff and my patronus is a Newfoundland :P.
\end{list2}



\end{resume}
\thispagestyle{lastpage}
\end{document}